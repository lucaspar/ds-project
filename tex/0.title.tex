% \title[Short Title]{Full Title}
\title[EnGendr -- Music Genre Classification]{EnGendr: an Ensemble Method for Music Genre Classification}
% I thought of this name + title in a minute, we can still change it.

\setreviewson
% \setreviewsoff

%% The "author" command and its associated commands are used to define
%% the authors and their affiliations.
%% Of note is the shared affiliation of the first two authors, and the
%% "authornote" and "authornotemark" commands
%% used to denote shared contribution to the research.

\author{Lucas Parzianello}
\affiliation{%
  \institution{University of Notre Dame}
  % \streetaddress{8600 Datapoint Drive}
  \city{Notre Dame}
  \state{Indiana}
  \postcode{46556}}
\email{lbarbosa@nd.edu}

% \author{Sophia Abraham}
% \affiliation{%
%   \institution{University of Notre Dame}
%   \city{Notre Dame}
%   \state{Indiana}
%   \postcode{46556}}
% \email{sabraha2@nd.edu}

\author{Eric Tsai}
\affiliation{%
  \institution{University of Notre Dame}
  \city{Notre Dame}
  \state{Indiana}
  \postcode{46556}}
\email{ctsai@nd.edu}

%%
%% By default, the full list of authors will be used in the page
%% headers. Often, this list is too long, and will overlap
%% other information printed in the page headers. This command allows
%% the author to define a more concise list
%% of authors' names for this purpose.
% \renewcommand{\shortauthors}{Parzianello, Abraham, and Tsai.}
\renewcommand{\shortauthors}{Parzianello and Tsai}

% UNCOMMENT:
% \begin{abstract}
%   \lipsum[9]
% \end{abstract}

%% Generate the section below at http://dl.acm.org/ccs.cfm.
% UNCOMMENT:
% \begin{CCSXML}
% <ccs2012>
% <concept>
% <concept_id>10010147.10010257</concept_id>
% <concept_desc>Computing methodologies~Machine learning</concept_desc>
% <concept_significance>300</concept_significance>
% </concept>
% <concept>
% <concept_id>10010147.10010178.10010224</concept_id>
% <concept_desc>Computing methodologies~Computer vision</concept_desc>
% <concept_significance>500</concept_significance>
% </concept>
% <concept>
% <concept_id>10010147.10010178.10010205</concept_id>
% <concept_desc>Computing methodologies~Search methodologies</concept_desc>
% <concept_significance>300</concept_significance>
% </concept>
% </ccs2012>
% \end{CCSXML}

% UNCOMMENT:
% \ccsdesc[300]{Computing methodologies~Machine learning}
% \ccsdesc[500]{Computing methodologies~Computer vision}
% \ccsdesc[300]{Computing methodologies~Search methodologies}

%%
%% Keywords. The author(s) should pick words that accurately describe
%% the work being presented. Separate the keywords with commas.

% UNCOMMENT:
% \keywords{COVID-19, machine learning benchmark}

%% A "teaser" image appears between the author and affiliation
%% information and the body of the document, and typically spans the
%% page.
% \begin{teaserfigure}
%   \includegraphics[width=\textwidth]{images/sampleteaser}
%   \caption{Seattle Mariners at Spring Training, 2010.}
%   \Description{Enjoying the baseball game from the third-base
%   seats. Ichiro Suzuki preparing to bat.}
%   \label{fig:teaser}
% \end{teaserfigure}

%%
%% This command processes the author and affiliation and title
%% information and builds the first part of the formatted document.
\maketitle
