\section{Possible Solutions}

Before getting into similar projects and potential solutions, we list below some of the features commonly found to solve tasks involving audio processing.

\subsection{Audio Features}

From the current literature, we have found the following two lists of handcrafted audio features with potential to aid an automated classifier. They are divided in time and frequency domains:

% interesting list:
% https://en.wikipedia.org/wiki/Category:Time%E2%80%93frequency_analysis

\subsubsection{Time domain}

\begin{itemize}
    \item Central Moments (CM)
    \item Zero Crossing Rate (ZCR)
    \item Root Mean Square Energy (RSME)
    \item Tempo
\end{itemize}

\subsubsection{Frequency domain}

\begin{itemize}
    \item Daubechies Wavelet Coefficient Histogram \source{Li}
    \item Mel-Frequency Cepstrum Coefficients (MFCC)
    \item Chroma Feature
    \item Spectral Centroid
    \item Spectral Bandwidth
    \item Spectral Contrast
    \item Sprectral Roll-Off
\end{itemize}

\subsection{Similar Projects}

\subsubsection{Common classification methods}

\todo[inline]{Write about RF, Logistic Regression, SVM, and Gradient Boosting in Bahuleyan's paper}
In \source{ Music Genre Classification using Machine Learning Techniques }, they explore the application of machine learning (ML) algorithms to identify and classify the genre of a given audio file. The conventional ML models used are Logistic Regression, Random Forests, Gradient Boosting, and Support Vector Machines(SVM).

\begin{itemize}
    \item Logistic Regression(LR): often used for binary classification tasks. For multi-class classification task, the LR is applied as a one-vs-rest approach.
    \item Random Forest(RF): a ensemble learner that combines the prediction from a pre-specified number of decision trees.
    \item Gradient Boosting (XGB):
    \item Support Vector Machines (SVM):
\end{itemize}

\subsubsection{Hierarchical Taxonomy}

\todo[inline]{Write about Li + Ogihara's paper}

\subsubsection{CNN}

\todo[inline]{Write about Zhang's "Improved" paper on CNN for music genre classification}

\subsection{Our Method}

\todo[inline]{Write about our solution(s) for this problem. Something along the following:}

The papers mentioned above attempt to solve the challenge of music genre classification, and show that \highlight{[ insights / takeaways about the problem ]}. However, they fail on \highlight{[ whatever our difference is ]}. To help with this challenge, we plan to \highlight{[ a high level description of our solution ]}
