\section{Preliminary Results}

\begin{figure*}
    \includesvg[width=0.8\textwidth]{images/baseline_accuracy.svg}
    \caption{Models vs. Features Sets and their Accuracy-measures on the validation set [0-100\%].}
    \label{fig:baseline_accuracy}
\end{figure*}

\begin{figure*}[tb!]
    \includesvg[width=0.8\textwidth]{images/baseline_f1}
    \caption{Models vs. Features Sets and their F1-measures on the validation set [0-100\%].}
    \label{fig:baseline_f1}
\end{figure*}



% \kant[1-10]     % remove me

% As shown in the , the features derived from MFCC combined with SVM classifiers provide the highest accuracy so far. The numbers will be more accurate if we train them with K-fold methods if we have enough time.

% \subsubsection{Analysis}
From the heatmap in Figure \ref{fig:baseline_accuracy}, we have found that the most decisive individual feature for genre classification across the classifiers tested is Mel-Frequency Cepstrum Coefficients (MFCC). Followed by the second most decisive feature is Spectral Contrast feature. The last rows of the heatmap combine these two features in sets with Spectral Contrast, Spectral Centroid, Chroma, Tonnetz and ZCR. When combined into feature sets, the experiments show a slight improvement on the accuracy and F1 measure on some classifiers, but a very similar performance to MFCC + Spectral Contrast overall. This performance is ratified by the F1 measure across different sets of features and classification models, shown in Figure \ref{fig:baseline_f1}. Finally, the combination of all features does not indicate a noteworthy improvement on the metrics.
