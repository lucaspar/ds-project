% PROJECT INSTRUCTIONS PDF:
%       https://sakailogin.nd.edu/access/content/group/FA20-CSE-40647-CX-01/Logistics/project_instruction-fa

% ==============
% FROM SYLLABUS:
% ==============

% Course Project Policies:
% Grading distribution
%       Proposal Paper (10%), Milestone Paper (30%), Final Presentation (20%)
%       Final Paper (30%), Code/Data (10%), Discussion bonus (+3%)
% Teaming: each graduate team should have 2 to 3 students.

% • Proposal
%   o Paper (PDF): title, problem definition, potential solutions, data sources,
%       proposed evaluation methods, project plan/timeline
% • Milestone
%   o Paper (PDF): introduction, related work, problem definition, one or more
%       method that has been used, some other potential solutions, data and
%       experiment settings, evaluation methods, preliminary results and
%       experimental analysis, timeline.
%   o Presentation (PPTX) for selective projects: in class
% • Final
%   o Paper (PDF): introduction, related work, problem definition,
%       solutions and methods that have been used, data and
%       experiment settings, evaluation methods, experimental
%       results and analysis (tables and figures), discussion and
%       future work, conclusions
%   o Presentation (PPTX): in class.
% • Data and code package (ZIP):
% 
% • The papers (PDF) should be formatted according to the new Standard ACM Conference
% Proceedings Template: https://www.acm.org/publications/proceedings-template
% 
% • There will be -33% penalty if the paper is not formatted correctly.
% 
% • Deadlines
%       Every project required item is due at midnight (11:55pm) with
%       some grace period. There will be -33% penalty for each 24h past the deadline.

% ========================= GRADING =========================

% Both milestone paper and final paper will be graded using the project paper rubric.
% Note that the ACM template adoption is a pre-assumed metric for all the papers.

% The project presentation will be graded as follows:
%   Introduction: 15% Provide context. What questions are being addressed?
%   Solution/Method: 30%
%       What did you do? Why did you choose this method? What tools and techniques
%       did you use?
%   Data and Experiments: 10% What data did you use? Are your experimental methods reliable?
%       Evaluation and Results: 30% What evaluation did you do? Do your conclusions match your results?
%   Presentation Quality: 15% Clarity of speaking (5%), organization (5%), and visuals (5%).

% The project paper will be graded as follows:
%   Introduction: 15%
%       Provide context and motivation. What questions are being addressed? Why are
%       these questions interesting or important?
%   Related Work: 10%
%       What other methods have addressed these or similar questions? How do these
%       methods differ from your method?
%   Solution/Method: 25%
%       What did you do? What tools and techniques did you use? Was any innovation
%       attempted?
%   Data and Experiments: 10%
%       What data did you use? Are your experimental methods reliable? What
%       preprocessing was done the data?
%   Evaluation and Results: 25%
%       Did you properly evaluate your experiments? Did you test for statistical
%       significance? Do your conclusions match your results?
%   Writing Quality: 15% Clarity of writing (5%), organization (5%), and grammar (5%).

% \documentclass[acmsmall,authordraft]{acmart}
\documentclass[sigconf]{acmart}

    \newcommand{\source}[1]{{\footnotesize \highlight{~\textsc{Source:}~#1~}}}
\newcommand{\name}[0]{Slotify}
\newcommand{\slotify}[0]{\name}

    \usepackage{lipsum}

% Easy Review:
% http://mirrors.ibiblio.org/CTAN/macros/latex/contrib/easyreview/doc/easyReview.pdf
\let\comment\undefined
\usepackage{easyReview}

% NOTE that a single column version is required for submission and peer review. This can be done by changing the \doucmentclass[...]{acmart} in this template to 
% \documentclass[manuscript,screen]{acmart}

%%
%% \BibTeX command to typeset BibTeX logo in the docs
\AtBeginDocument{%
  \providecommand\BibTeX{{%
    \normalfont B\kern-0.5em{\scshape i\kern-0.25em b}\kern-0.8em\TeX}}}

%% Rights management information.  This information is sent to you
%% when you complete the rights form.  These commands have SAMPLE
%% values in them; it is your responsibility as an author to replace
%% the commands and values with those provided to you when you
%% complete the rights form.
% \setcopyright{acmcopyright}
% \copyrightyear{2018}
% \acmYear{2018}
% \acmDOI{10.1145/1122445.1122456}

%% These commands are for a PROCEEDINGS abstract or paper.
% \acmConference[Woodstock '18]{Woodstock '18: ACM Symposium on Neural
%   Gaze Detection}{June 03--05, 2018}{Woodstock, NY}
% \acmBooktitle{Woodstock '18: ACM Symposium on Neural Gaze Detection,
%   June 03--05, 2018, Woodstock, NY}
% \acmPrice{15.00}
% \acmISBN{978-1-4503-XXXX-X/18/06}

%%
%% Submission ID.
%% Use this when submitting an article to a sponsored event. You'll
%% receive a unique submission ID from the organizers
%% of the event, and this ID should be used as the parameter to this command.
%%\acmSubmissionID{123-A56-BU3}

%%
%% The majority of ACM publications use numbered citations and
%% references.  The command \citestyle{authoryear} switches to the
%% "author year" style.
%%
%% If you are preparing content for an event
%% sponsored by ACM SIGGRAPH, you must use the "author year" style of
%% citations and references.
%% Uncommenting
%% the next command will enable that style.
%%\citestyle{acmauthoryear}

    
    \begin{document}
    
        % \title[Short Title]{Full Title}
\title[EnGendr -- Music Genre Classification]{EnGendr: an Ensemble Method for Music Genre Classification}
% I thought of this name + title in a minute, we can still change it.

\setreviewson
% \setreviewsoff

%% The "author" command and its associated commands are used to define
%% the authors and their affiliations.
%% Of note is the shared affiliation of the first two authors, and the
%% "authornote" and "authornotemark" commands
%% used to denote shared contribution to the research.

\author{Lucas Parzianello}
\affiliation{%
  \institution{University of Notre Dame}
  % \streetaddress{8600 Datapoint Drive}
  \city{Notre Dame}
  \state{Indiana}
  \postcode{46556}}
\email{lbarbosa@nd.edu}

% \author{Sophia Abraham}
% \affiliation{%
%   \institution{University of Notre Dame}
%   \city{Notre Dame}
%   \state{Indiana}
%   \postcode{46556}}
% \email{sabraha2@nd.edu}

\author{Eric Tsai}
\affiliation{%
  \institution{University of Notre Dame}
  \city{Notre Dame}
  \state{Indiana}
  \postcode{46556}}
\email{ctsai@nd.edu}

%%
%% By default, the full list of authors will be used in the page
%% headers. Often, this list is too long, and will overlap
%% other information printed in the page headers. This command allows
%% the author to define a more concise list
%% of authors' names for this purpose.
% \renewcommand{\shortauthors}{Parzianello, Abraham, and Tsai.}
\renewcommand{\shortauthors}{Parzianello and Tsai}

% UNCOMMENT:
% \begin{abstract}
%   \lipsum[9]
% \end{abstract}

%% Generate the section below at http://dl.acm.org/ccs.cfm.
% UNCOMMENT:
% \begin{CCSXML}
% <ccs2012>
% <concept>
% <concept_id>10010147.10010257</concept_id>
% <concept_desc>Computing methodologies~Machine learning</concept_desc>
% <concept_significance>300</concept_significance>
% </concept>
% <concept>
% <concept_id>10010147.10010178.10010224</concept_id>
% <concept_desc>Computing methodologies~Computer vision</concept_desc>
% <concept_significance>500</concept_significance>
% </concept>
% <concept>
% <concept_id>10010147.10010178.10010205</concept_id>
% <concept_desc>Computing methodologies~Search methodologies</concept_desc>
% <concept_significance>300</concept_significance>
% </concept>
% </ccs2012>
% \end{CCSXML}

% UNCOMMENT:
% \ccsdesc[300]{Computing methodologies~Machine learning}
% \ccsdesc[500]{Computing methodologies~Computer vision}
% \ccsdesc[300]{Computing methodologies~Search methodologies}

%%
%% Keywords. The author(s) should pick words that accurately describe
%% the work being presented. Separate the keywords with commas.

% UNCOMMENT:
% \keywords{COVID-19, machine learning benchmark}

%% A "teaser" image appears between the author and affiliation
%% information and the body of the document, and typically spans the
%% page.
% \begin{teaserfigure}
%   \includegraphics[width=\textwidth]{images/sampleteaser}
%   \caption{Seattle Mariners at Spring Training, 2010.}
%   \Description{Enjoying the baseball game from the third-base
%   seats. Ichiro Suzuki preparing to bat.}
%   \label{fig:teaser}
% \end{teaserfigure}

%%
%% This command processes the author and affiliation and title
%% information and builds the first part of the formatted document.
\maketitle

    
        % PROPOSAL:
        \section{Introduction}

\lipsum[2-6]

        \section{Related Work}

\lipsum[2]

        \section{Data Sources}

From our research, we have found a quite few options of datasets containing music tracks (excerpts or integral) with music genre labels. Some popular alternatives are described below:

\subsection{GTZAN}

GTZAN is one of the most popular public datasets for music genre recognition \cite{Sturm2013} and is composed of a thousand 30-second audio excerpts labeled across 10 music genres. Despite its popularity, the dataset was not created for music genre classification. Moreover, there are many critics about the dataset quality and whether its size is capable of allowing for accurate or significant results \cite{Sturm2013}.

\subsection{SYNAT}

The SYNAT database \cite{10.1007/978-3-642-21916-0_75} stores over 50 thousand 30-second music tracks in MP3 format, across 22 genres: Alternative Rock, Blues, Broadway \& Vocalists, Children’s Music, Christian and Gospel, Classic Rock, Classical, Country, Dance and DJ, Folk, Hard Rock and Metal, International, Jazz, Latin Music, Miscellaneous, New Age, Opera \& Vocal, Pop, Rap and Hip-Hop, Rock, R\&B, and Soundtracks. However, we were unable to find a working download link or request form at the time of writing.

\subsection{MSD}

The Million Song Dataset (MSD) \cite{Bertin-Mahieux2011} is a collection of one million songs for which over 190 thousand tracks have consistent genre annotations. Due to the large size of the dataset (around 300GB), MSD is publicly available for research purposes as an AWS EC2 snapshot, rather than a direct download.

\subsection{FMA}

The Free Music Archive (FMA) dataset \cite{Defferrard2017} is a publicly available alternative containing over 100 thousand audio tracks with four dataset versions of varying track number, lengths, and genres, ranging from 8 thousand tracks of 30 seconds of 7.2GB in total size; to over 106 thousand untrimmed tracks across 161 genres summing 879GB. The audio tracks are under a Creative Commons license and it appears to be the best documented alternative.

Due to the public availability, ease of access, and good documentation, we are inclined to start experimenting with a subset of the FMA dataset.

        \section{Evaluation}

In order to compare the classification models and fulfill our contributions of (i) which features are most relevant in the handcrafted method, and (ii) build an ensemble model for music genre classification; we plan to extract a list of metrics from our classifiers. Firstly, our dataset will be split into training, validation, and testing sets with disjoint audio tracks and uniform representation across music genres, when possible. Then, once the classifiers models are built, we will extract the metrics below in isolation, and lastly, from our ensemble version.

\subsection{Metrics}

Some of the metrics we are considering using are:

\begin{itemize}
    \item Mean accuracy;
    \item F1 score;
    \item Confusion matrix;
    \item Area under the ROC Curve (AUC); % if any thresholding is applied
    % Q:how to determine which features are most relevant?
\end{itemize}

        \section{Data Processing}

% \subsection{Data Cleaning}

From the FMA-Small dataset we select \texttt{trackid} as our object and \texttt{genre-top} as our ground truth. FMA-Small dataset contains 8000 tracks in 8 different genre categories, which are:
'Hip-Hop', 'Pop', 'Folk', 'Experimental', 'Rock', 'International', 'Electronic', and 'Instrumental'.

% FMA-Small 8 genres:
% 'Hip-Hop', 'Pop', 'Folk', 'Experimental', 'Rock', 'International', 'Electronic', and 'Instrumental'.

% FMA-Medium 16 genres:
% Blues, Classical, Country, Easy Listening, Electronic, Experimental, Folk, Hip-Hop, Instrumental, International, Jazz, Old-Time Historic, Pop, Rock, Soul-RnB, and Spoken.


\subsection{Feature Extraction}

We are currently using the following features:

\begin{itemize}
    \item Zero Crossing Rate (ZCR) \cite{Li2006}
    \item Root Mean Square Energy (RSME) \cite{Tao}
    \item Mel-Frequency Cepstrum Coefficients (MFCC) \cite{Li2006, Nanni2016, Hoffmann2016, Lim2012}
    \item Spectral Centroid, Bandwidth, Contrast, Roll-Off \cite{Li2006, Li2005}
    \item Chroma Features
    \item Tonal Centroid Features (Tonnetz) \cite{Harte2006}
\end{itemize}

In addition to those, the features that we have found in the literature but are currently not extracted:

\begin{itemize}
    \item Daubechies Wavelet Coefficient Histogram \cite{Li2006}
    \item Central Moments (CM)
    \item Tempo
\end{itemize}

        \section{Experiment Settings}

For training purposes, we split the FMA-Medium dataset into 8:1:1, respectively training, validation, and testing.

\subsection{Models}
After the feature extraction, we make them as input and send them to following models for training process:
\begin{itemize}
    \item Logistic Regression (LR)
    \item Random Forests (RF)
    \item K-nearest Neighbors (KNN)
    \item Support Vector Classification (SVC)
    \item DecisionTreeClassifier (DT)
    \item AdaBoostClassifier (AdaBoost)
    \item GaussianNB (NB)
    \item QuadraticDiscriminantAnalysis (QDA)
\end{itemize}

        \section{Preliminary Results}

\begin{figure*}[ht]
    \includesvg[width=0.8\textwidth]{images/baseline_accuracy}
    \caption{Models vs. Feature Sets and their accuracies on the validation set [0-100\%].}
    \label{fig:baseline_accuracy}
\end{figure*}

\par~

\begin{figure*}[ht]
    \includesvg[width=0.8\textwidth]{images/baseline_f1}
    \caption{Models vs. Features Sets and their F1-measures on the validation set [0-100\%].}
    \label{fig:baseline_f1}
\end{figure*}

% \kant[1-10]     % remove me

% As shown in the , the features derived from MFCC combined with SVM classifiers provide the highest accuracy so far. The numbers will be more accurate if we train them with K-fold methods if we have enough time.

% \subsubsection{Analysis}

From the heatmap in Figure \ref{fig:baseline_accuracy}, we have found that the most decisive individual feature for genre classification across the classifiers tested is Mel-Frequency Cepstrum Coefficients (MFCC). Followed by the second most decisive feature is Spectral Contrast feature. The last rows of the heatmap combine these two features in sets with Spectral Contrast, Spectral Centroid, Chroma, Tonnetz and ZCR. When combined into feature sets, the experiments show a slight improvement on the accuracy and F1 measure on some classifiers, but a very similar performance to MFCC + Spectral Contrast overall. This performance is ratified by the F1 measure across different sets of features and classification models, shown in Figure \ref{fig:baseline_f1}. Finally, the combination of all features does not indicate a noteworthy improvement on the metrics.

        \section{Timeline}

With synchronization checkpoints every Monday, we have reached the fourth week as planned in our proposal:

\begin{table}[h]
    \begin{tabular}{lcl}
    \textbf{Date}    & \multicolumn{1}{l}{\textbf{Week}} & \textbf{Tasks}                                        \\ \hline
    \st{Sep. 14} & \st{1}                        & \st{Dataset selection, download, and cleanup.}                \\
    \st{Sep. 21} & \st{2}                        & \st{Handcrafted feature extraction.}                          \\
    \st{Sep. 28} & \st{3}                        & \st{Implement baseline classifier; milestone.} \\
    Oct. 5  & 4   & Improvements on baseline; metrics extraction. \\
    Oct. 12 & 5   & Alternative model implementation.             \\
    Oct. 19 & 6   & New metrics and comparison.                   \\
    Oct. 27 & 7   & Final report writing and presentation.        \\
    Nov. 2  & 8   & Further experiments / improvements.           \\
    Nov. 9  & 8.5 & Revision and delivery.                       
    \end{tabular}
\end{table}

The next steps include extracting further metrics by running the existing classifiers on FMA-Medium. Following that, we will combine the best classifiers into a new ensemble version and tune their hyperparameters.

        
        % MILESTONE:
        % \section{Introduction}

\lipsum[2-6]

        % \section{Related Work}

\lipsum[2]

        % \section{Problem Definition}

\lipsum[3]

        % \section{Methodology}

\lipsum[4]

        % \section{Experiments}

\lipsum[5]

        % \input{tex/6.evaluation}
        % \section{Preliminary Results}

\begin{figure*}[ht]
    \includesvg[width=0.8\textwidth]{images/baseline_accuracy}
    \caption{Models vs. Feature Sets and their accuracies on the validation set [0-100\%].}
    \label{fig:baseline_accuracy}
\end{figure*}

\par~

\begin{figure*}[ht]
    \includesvg[width=0.8\textwidth]{images/baseline_f1}
    \caption{Models vs. Features Sets and their F1-measures on the validation set [0-100\%].}
    \label{fig:baseline_f1}
\end{figure*}

% \kant[1-10]     % remove me

% As shown in the , the features derived from MFCC combined with SVM classifiers provide the highest accuracy so far. The numbers will be more accurate if we train them with K-fold methods if we have enough time.

% \subsubsection{Analysis}

From the heatmap in Figure \ref{fig:baseline_accuracy}, we have found that the most decisive individual feature for genre classification across the classifiers tested is Mel-Frequency Cepstrum Coefficients (MFCC). Followed by the second most decisive feature is Spectral Contrast feature. The last rows of the heatmap combine these two features in sets with Spectral Contrast, Spectral Centroid, Chroma, Tonnetz and ZCR. When combined into feature sets, the experiments show a slight improvement on the accuracy and F1 measure on some classifiers, but a very similar performance to MFCC + Spectral Contrast overall. This performance is ratified by the F1 measure across different sets of features and classification models, shown in Figure \ref{fig:baseline_f1}. Finally, the combination of all features does not indicate a noteworthy improvement on the metrics.

        % \section{Timeline}

With synchronization checkpoints every Monday, we have reached the fourth week as planned in our proposal:

\begin{table}[h]
    \begin{tabular}{lcl}
    \textbf{Date}    & \multicolumn{1}{l}{\textbf{Week}} & \textbf{Tasks}                                        \\ \hline
    \st{Sep. 14} & \st{1}                        & \st{Dataset selection, download, and cleanup.}                \\
    \st{Sep. 21} & \st{2}                        & \st{Handcrafted feature extraction.}                          \\
    \st{Sep. 28} & \st{3}                        & \st{Implement baseline classifier; milestone.} \\
    Oct. 5  & 4   & Improvements on baseline; metrics extraction. \\
    Oct. 12 & 5   & Alternative model implementation.             \\
    Oct. 19 & 6   & New metrics and comparison.                   \\
    Oct. 27 & 7   & Final report writing and presentation.        \\
    Nov. 2  & 8   & Further experiments / improvements.           \\
    Nov. 9  & 8.5 & Revision and delivery.                       
    \end{tabular}
\end{table}

The next steps include extracting further metrics by running the existing classifiers on FMA-Medium. Following that, we will combine the best classifiers into a new ensemble version and tune their hyperparameters.

        
        \clearpage
        \bibliographystyle{bib/ACM-Reference-Format}
\bibliography{bib/papers}

        % \input{tex/_appendix}
    
    \end{document}
\endinput
